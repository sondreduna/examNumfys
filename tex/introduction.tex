The topic of this project is epidemic modelling, and we will study various different models. A short introduction to these are given in this introduction, mostly for introducing notation and presenting some choices in the implementation. For a more detailed introduction, consult the problem sheet \cite{sheet}.

\subsection{The deterministic SIR model}
The SIR model reads 
\begin{subequations}\label{eq:SIR}
\begin{align}
	\der{S}{t} &= - \beta \frac{I S}{N}\\
	\der{I}{t} &= \beta \frac{IS }{N} - I/\tau \\
	\der{R}{t} &= I/\tau, 
\end{align}
\end{subequations}
where $\beta$ is the rate at which susceptible people encounter infected people, and become infected, and $\tau$ is the typical duration of the infection. To solve this equation numerically, we recast it as a system of ODE's by defining the vector $\mathbf{v} = [S,I,R]^T$, such that the equations in \ref{eq:SIR} simplify to
\begin{equation}\label{eq:sir_vect}
	\der{\mathbf{v}(t)}{t} = \frac{\mathrm{d}}{\mathrm{d}t} \begin{bmatrix}
		S(t) \\
		I(t) \\
		R(t)
	\end{bmatrix}
	= \begin{bmatrix}
		- \beta IS/N \\
		\beta  IS/N - I/\tau \\
		I/\tau
	\end{bmatrix} = \begin{bmatrix}
		-\beta v_1 v_2 /N \\
		\beta v_1 v_2 - v_2/\tau \\
		v_1 / \tau
	\end{bmatrix}= \mathbf{f}(t,\mathbf{v}).
\end{equation}

\subsection{The stochastic SIR model}

For the stochastic SIR model we have probabilities instead of rates. The probabilities for a susceptible person becoming infected, and an infected person recovering during an interval of $\Delta t$ are given by
\begin{subequations}\label{eq:Psir}
	\begin{align}
		P_{S\to I} &= 1- \exp{(-\Delta t \beta I /N)} \\
		P_{S\to R} &= 1- \exp{(-\Delta t /\tau)},
	\end{align}
\end{subequations}
respectively. The transition to next time step is then governed by 
\begin{subequations}\label{eq:sir_disc}
	\begin{align}
		S(t + \Delta t) &= S(t) - \Delta_{S\to I} \\
		I(t + \Delta t) &= I(t) + \Delta_{S\to I} - \Delta_{I\to R} \\
		R(t + \Delta t) &= R(t) + \Delta_{I \to R}, \\
	\end{align}
\end{subequations}
where the $\Delta$'s are drawn from binomial distributions:
\begin{subequations}
	\begin{align*}
		\Delta_{S\to I } &= \mathcal{B}(S,P_{S \to I}) \\
		\Delta_{I\to R } &= \mathcal{B}(I,P_{I \to R}).
	\end{align*}
\end{subequations}
To see the connection between this model and the deterministic one, suppose $\Delta t \times \beta \ll 1$, and $\Delta t/\tau \ll 1$. In that case, the probabilities in equation \eqref{eq:Psir} are \textit{almost} 0, 
\begin{subequations}
	\begin{align}
		P_{S\to I} &\simeq \Delta t \beta I/N \\
		P_{I\to R} &\simeq \Delta t/\tau. 
	\end{align}
\end{subequations}
In the limit where $n\to \infty$\footnote{More formally, $n\to \infty$ and $p\to 0$ in such a fashion that the product $np$ is finite. }, the binomial distribution with $n$ samples and probability $p$ converges to a Poisson distribution with $\lambda = np$ \cite{Wassermann}, i.e. 
\begin{subequations}
	\begin{align}
		\Delta_{S\to I} &\to \mathrm{Pois}(SI \Delta t \beta/N) \\
		\Delta_{I\to R} &\to \mathrm{Pois}(I \Delta t/\tau).
	\end{align}
\end{subequations}
As the pdf of the Poisson-distribution is given by $p(k\vert \lambda) = \lambda^k \mathrm{e}^{-\lambda}/k!$, we can find expressions for the Delta's if we assume them to attain the mean value of the Possion distributions, namely $\lambda$:
\begin{subequations}
	\begin{align}
		\Delta_{S\to I} &\simeq SI \Delta t \beta/N  \\
		\Delta_{I\to R} &\simeq I \Delta t/\tau.
	\end{align}
\end{subequations}
Hence, the discrete transition equations \eqref{eq:sir_disc} are 
\begin{subequations}\label{eq:sir_disc}
	\begin{align}
		S(t + \Delta t) &\simeq S(t) - SI \Delta t \beta/N\\
		I(t + \Delta t) &\simeq I(t) + SI \Delta t \beta/N - I \Delta t/\tau \\
		R(t + \Delta t) &\simeq R(t) + I \Delta t/\tau, \\
	\end{align}
\end{subequations}
which yields the same equations as the deterministic ones in \eqref{eq:SIR}, by formally taking the limit $\Delta t \to 0$.

\subsection{The stochastic SEIIaR model}

For the stochastic SEIIaR model we follow essentially the same line of thought as for the stochastic SIR model, except that we include $E$ : exposed people, that cannot infect others, and $I_a$ : infected but asymptomatic people. The probabilities for the different transitions happening during an interval $\Delta t$ are
\begin{subequations}
	\begin{align}
		P_{S\to E} &= 1- \exp{\left(-\Delta t \beta \frac{r_s I + r_a I_a}{N}\right)} \\
		P_{E\to I} &=  f_s \times (1- \exp{(-\Delta t /\tau_E)}) \\
		P_{E\to I_a} &= f_a \times (1- \exp{(-\Delta t/\tau_E)}) \\
		P_{I\to R} &= 1- \exp{(-\Delta t /\tau_I)} \\
		P_{I_a\to R} &= 1- \exp{(-\Delta t/\tau_I)}, 
	\end{align}
\end{subequations}
with the different parameters $r_a,r_s,f_a,f_s,\tau_E,\tau_I$ given in \cite{sheet}. As before, 
\begin{subequations}
	\begin{align*}
		\Delta_{S\to E  } &= \mathcal{B}(S,P_{S \to E}) \\
		\Delta_{I\to R  } &= \mathcal{B}(I,P_{I \to R}) \\
		\Delta_{I_a\to R} &= \mathcal{B}(I_a,P_{I_a \to R}).
	\end{align*}
\end{subequations}
and 
\begin{equation}
	\Delta_{E \to I}, \Delta_{E \to I_a}, \Delta_{E \to E} = \mathcal{M}(E,(P_{E\to I},P_{E\to I_a}, (1- P_{E\to I} - P_{E\to I_a}))),
\end{equation}
where $\mathcal{M}$ denotes a multinomial distribution, as described in \cite{sheet}. The changes of the different variables during an interval $\Delta t$ are then given by
\begin{subequations}
	\begin{align}
		S(t + \Delta t) &= S(t) - \Delta_{S\to E} \\
		E(t + \Delta t) &= E(t) + \Delta_{S\to E} - \Delta_{E\to I} - \Delta_{E \to I_a} \\
		I(t + \Delta t) &= I(t) + \Delta_{E\to I} - \Delta_{I\to R} \\
		I_a(t + \Delta t) &= I_a(t) + \Delta_{E\to I_a} - \Delta_{I_a\to R} \\
		R(t + \Delta t) &= R(t) + \Delta_{I_a \to R} + \Delta_{I \to R}. \\
	\end{align}
\end{subequations}
To denote a state of the system, I will use $\mathbf{X} = [S,E,I,I_a,R]^T$. 

\vspace{2cm}