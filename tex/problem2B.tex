\section{Problem 2B: Stochastic SIR model}

\subsection{a)}

As seen in the plot in figure \ref{fig:SIR_stoch}, the $10$ different realisations of the stochastic SIR model , shown in dashed lines in graded colours, seem to lie close to the deterministic solution.

\begin{figure}[htb]
	\centering
	\includegraphics[width=0.8\columnwidth]{../fig/2Ba_SIR.pdf}
	\caption{Solution of stochastic SIR equations with $\beta = 0.25\, \mathrm{day}^{-1}$, $\tau = 10\, \mathrm{day}$.}
	\label{fig:SIR_stoch}
\end{figure}

\subsection{b)}

As in the previous exercise, we plot the fraction of infected people together with the analytical model for the early stages. This is shown in figure \ref{fig:Infected_stoch}. Here we clearly see that all the realisations are approximately linear in the semi-log plot in the first $40$ days, or so, of the simulation as we observed in the previous exercise. 

\begin{figure}[htb]
	\centering
	\includegraphics[width=0.8\columnwidth]{../fig/2Bb_I.pdf}
	\caption{Infected people compared with the analytical approximation at the early stages. Stochastic and deterministic model.}
	\label{fig:Infected_stoch}
\end{figure}

\subsection{c) Probability of an outbreak}

There will always be a certain probability for an outbreak disappearing by itself for the stochastic model. In the present subsection we estimate this probability for an initial number of infected people $=1,2,\dots,10$. This is done by the following procedure:

\begin{algorithm}[H]
	Choose the parameters of the model as those given in the exam sheet \cite{sheet}, but with $T = 30 \, \mathrm{days}$\footnote{As the typical infection time is $10$ days, I assume 50 days to be sufficient for detecting an outbreak in the stochastic model.}. \;
	Choose a batch-size $B$.\;
	\For{$I = 1,2,\dots, 10$}{
		Initialise an empty vector of length $B$: $\mathbf{X} = [0,\dots,0]$.\;
		\For{$n = 1,\dots, B$}
			{
			Run the simulation with initial number of infected people $= I$.\;
			Calculate the \texttt{slope} in the semi-log axes for $I(t)$.\;
			\eIf{$\texttt{slope} <= 0$}
			{$X_n = 0$}
			{$X_n = 1$}
		}
		Estimate the probability of an outbreak for $I$ initially infected by 
		$$
				p \coloneqq P(\mathrm{outbreak}|I) = \frac{1}{B} \sum_{n= 1}^{B} X_n. \;
		$$	
		Calculate the standard deviation of the estimate by 
		\begin{equation}\label{eq:std}
			\sqrt{\mathrm{Var}(\hat{p})} = \sqrt{\frac{p(1-p)}{B}}.
		\end{equation}
	}
	\caption{Calculating the probability of an outbreak as a function of the initial number of infected people, $I$. }
\end{algorithm} 

In performing this procedure, I use a batch size of $500$ in each sweep. What we are estimating here is essentially a Bernoulli-distributed random variable, $X$: $X$ can take the realisations $1$ or $0$ with probabilities $p$ and $1-p$ respectively, and we assume them to be independent \cite[~p.26]{Wassermann}. Then, as the variance of such a distribution is $p(1-p)$, the variance of the estimator for the probability, namely $\hat{p}$ i.e. the expectation value of $X_n$, is 
$$
	\mathrm{Var}(\hat{p}) =  \sum_{n= 1}^{B} \mathrm{Var}\left(\frac{X_n}{B}\right) = \frac{1}{B^2} \sum_{n= 1}^{B} \mathrm{Var}(X_n) = \frac{1}{B^2} \sum_{n= 1}^{B} p (1-p) = \frac{p(1-p)}{B},
$$
from which formula \eqref{eq:std} follows. 

The probability of an outbreak as a function of the initial number of infected people are shown in figure \ref{fig:prob_outbreak} together with the associated standard deviation. As seen from this plot, when there are more than $6$ people initially infected, the probability is approximately $1$ that an outbreak will happen. However, for e.g. $1$ initially infected person, the probability is less than $0.6$. This ultimately shows that the stochastic model has a more realistic feature to it than the deterministic one, in that these scenarios might occur.
% The probabilities are also shown in table \ref{tab: probabilities}

\begin{figure}[htb]
	\centering
	\includegraphics[width=0.8\columnwidth]{../fig/2Bc_prob.pdf}
	\caption{Probability of an outbreak as a function of initial number of infected people.}
	\label{fig:prob_outbreak}
\end{figure}

\clearpage